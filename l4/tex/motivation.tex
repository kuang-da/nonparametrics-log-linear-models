\section{Motivation of Nonparametric Methods}
Traditional parametric testing methods are based on the
assumption that data are generated by well-known distributions,
characterized by one or more unknown parameters.

The critical values or alternatively the $p$-values is computed
according to the distribution of the test statistic under the null
hypothesis, which can be derived from the assumptions related
to the assumed underlying distribution of data.

But the above assumption is not always true. For example, if the data is semi-continuous and also zero inflated, then any assumed distribution is not true. We can still apply the parametric testing method on the data but the result is probably not reasonable.

When the parametric does not hold, nonparametric methods are the valid solutions. Nonparametric methods require relatively mild assumptions
regarding the underlying populations from which the data are
obtained.

Note that when the parametric assumptions hold, the nonparametric
methods are only slightly less powerful than the parametric
methods.


