\section{Matching}
\subsection{Motivation}
In an observational study, subjects in the treatment group may
not be comparable to those in the control group. Differences in
the outcome between the groups are then not necessarily
attributable to a treatment effect but instead may arise from
differences in confounders.

A popular approach to controlling for
confounders is to construct matched pairs through optimal
multivariate matching (Paul R. Rosenbaum. Design of Observational Studies).

\subsection{Eliminating the Confounding by Matching}
Matching is usually performed by matching a fixed ratio of
controls to treatment subjects, most commonly pair (one-to-one)
matching. Sometimes one-to-two or one-to-three matching. But we rarely have more than one-to-three matching. It is because that the computation cost increase but only gives us marginal improvement.Alternative forms of matching may bring desirable features such as variable ratio matching and full matching.

For instance, suppose we have 100 treatment patient and 200 control patients, then we select 100 control out of the 200 to match treatment patients. We usually only focus on the matched groups but sometime it is beneficial to do some extra calculation on the unmatched group. The unmatched subjects can also be used to reduce confounding
bias.

After constructing the match, we check if the observed
covariates are balanced so that they have similar distributions,
i.e., whether the observed confounders are balanced.

The absolute standardized difference ($|D|$) is usually used to
check covariate balance and evaluate the match quality.

The numerator of $|D|$ is the absolute difference between the
treated and the (matched) control in covariate means, and the
denominator is the pooled standard deviation before the match,
the square root of the average of the two groups' sample
variances. Smaller absolute standardized difference implies closer
covariate proximity.

For instance, a $|D| < 0.1$ indicates that the absolute mean
difference is within $10\%$ of the pooled standard deviation.

\subsection{Assess the Matching By Hypothesis Testing}
To assess covariate balance, formal tests can also be used,
namely the Wilcoxon rank sum test for continuous variables and
Fisher's exact test for binary variables.

If the observed confounders are balanced, then, assuming that
there are no unmeasured confounders, outcome analysis can be
constructed in straightforward ways as in a matched pair
randomized trial.

McNemar's test can be used for binary outcomes.

Wilcoxon signed rank test for continuous outcomes.
