\section{Properties of Odds Ratio}
\subsection{Directed Association}
The odds ratio $\theta$ measures the strength of association between $X$ and $Y$.
\begin{itemize}
	\item $1 < \theta < \infty$ implies that the odds of $Y = 1$, given that $X = 1$, is larger than the odds that $Y = 1$, given that $X = 0$.
	\item $0 < \theta < 1$ implies that the odds of $Y = 1$, given that $X = 1$, is smaller than the odds that $Y = 1$, given that $X = 0$.
\end{itemize}

Values of $\theta$ farther from 1 in a given direction represent stronger association.

Note that the association is directed.

\begin{itemize}
	\item The odd ratio does not change value when the table orientation reverses so that the rows become the columns and the columns become the rows.
	\item Two values for $\theta$ represent the same strength of association, but
in opposite directions, when one value is the inverse of the other.
\end{itemize}

\subsection{Example: $\theta = 4$ vs. $\theta = 1/4$}
\begin{itemize}
	\item When the order of the rows is reversed or the order of the
	columns is reversed, the new value of $\theta$ is the inverse of the
	original value.
	\item This ordering is usually arbitrary, so whether we get 4 or 0.25 for
	the odds ratio is merely a matter of how we label the rows and
columns.
\end{itemize}

\subsection{Equivalent to independence}
The odd ratio $\theta = 1$ $\iff$ $X$ and $Y$ are independent.
\begin{proof}
	\begin{itemize}
		\item If $X$ and $Y$ are independent, then
		\begin{align*}
			\theta_1 &= \frac{\P(Y = 1| X = 1)}{P(Y = 0| X = 1)} = \frac{\P(Y = 1)}{\P(Y = 0)}\\
			\theta_2 &= \frac{\P(Y = 1| X = 0)}{P(Y = 0| X = 0)} = \frac{\P(Y = 1)}{\P(Y = 0)}\\
			\theta &= \frac{\theta_1}{\theta_2} = 1
		\end{align*}
		\item If $\theta = 1$, then
		\begin{align*}
			\frac{\P(Y = 1, X = 1)}{\P(Y = 0, X = 1)} 
			&= \frac{\P(Y = 1, X = 0)}{\P(Y = 0, X = 0)} \\
			\frac{\P(Y = 1, X = 1)}{\P(X = 1) - \P(Y = 1, X = 1)} 
			&= \frac{\P(Y = 1, X = 0)}{\P(X = 0) - \P(Y = 1, X = 0)} \\
			\frac{\P(Y = 1, X = 1)}{\P(X = 1) - \P(Y = 1, X = 1)} 
			&= \frac{\P(Y = 1, X = 0)}{1 - \P(X = 1) - \P(Y = 1, X = 0)} \\
			\P(X = 1, Y = 1) &= \P(X = 1)\P(Y = 1)
		\end{align*}
	Hence, we have $X$ and $Y$ are independent.
	\end{itemize}
\end{proof}

Therefore the null hypothesis about independence can be reduced to the odd ratio $\theta = 1$.

\subsection{Example: Aspirin vs Cardiovascular Disease}

The Physicians' Health Study was a five-year randomized study
testing whether regular intake of aspirin reduces mortality from
cardiovascular disease. Every other day, the male physicians
participating in the study took either one aspirin tablet or a placebo.
The study was ``blind'' --- the physicians in the study did not know
which type of pill they were taking.

\begin{figure}[H]
	\centering
	\includegraphics[width=0.7\linewidth]{fig/screenshot003}
	\caption{Preliminary Report: Findings from the Aspirin Component of the Ongoing Physicians' Health Study. New Engl. J. Med., 318: 262-264, 1988.}
	\label{fig:screenshot003}
\end{figure}

\[\theta = \frac{189 \times 10933}{104 \times 10845} = 1.83\]

The estimated odds of Myocardial Infarction for male physicians taking placebo equals
1.83 times the estimated odds for male physicians taking aspirin. The
estimated odds were $83\%$ higher for the placebo group.

\subsection{Large Sample Inference}
If the sample size is large enough, we can use central limit theorem (CLT) for hypothesis testing and estimate the confidence interval.

Note that the range of odd ratio for association is not symmetric, $0 < \theta < 1$ and $1 < \theta < \infty$. So before applying CLT, we need to transform the odd ratio into log-space, i.e. $\lambda = \ln \theta$. 

Then the sample estimation will be $\hat{\lambda} = \ln \hat{\theta}$. The estimated standard deviation is $\hat{\text{SD}} (\hat{\lambda}) = \sqrt{\frac{1}{n_{11}} +\frac{1}{n_{12}} + \frac{1}{n_{21}} + \frac{1}{n_{22}}}$.

The null hypothesis of indepence, $\theta = 1$ can be reduced to $\lambda = 0$.

We can apply CLT to compare $\hat{\lambda}/\hat{\text{SD}} (\hat{\lambda})$ with normal quantile for hypothesis.

Two-sided confidence interval:
\[
	\hat{\lambda} - z_{\alpha/2} \hat{\text{SD}}(\hat{\lambda}) 
	\le \lambda \le
	\hat{\lambda} + z_{\alpha/2} \hat{\text{SD}}(\hat{\lambda}) 
\]

Transform to the odd ratio:
\[
	\exp{(\hat{\lambda} - z_{\alpha/2} \hat{\text{SD}}(\hat{\lambda}))} 
	\le \theta \le
	\exp{(\hat{\lambda} + z_{\alpha/2} \hat{\text{SD}}(\hat{\lambda}))} 
\]
