\section{Relative Risk}
A difference between two proportions of a certain fixed size usually is
more important when both proportions are near 0 or 1 than when
they are near the middle of the range.

Consider a comparison of two drugs on the proportion of
subjects who had adverse reactions when using the drug.
\begin{itemize}
	\item The difference between 0.010 and 0.001 is the same as the
difference between 0.410 and 0.401, namely 0.009.
	\item The first difference is more striking, since 10 times as many
subjects had adverse reactions with one drug as the other.
\end{itemize}

In such cases, the ratio of proportions, namely relative risk, is a more relevant descriptive measure. For the above example,

\begin{itemize}
	\item The proportions 0.010 and 0.001 have a relative risk of 0.010/0.001 = 10.
	\item The proportions 0.410 and 0.401 have a relative risk of 0.010/0.001 = 1.02.
\end{itemize}

A relative risk of 1.00 occurs when $p_1 = p_2$, that is when the response is independent of the group.
