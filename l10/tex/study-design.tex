\section{Review: Study Design}
\subsection{Conditional Design}
When the rows of a contingency table refer to different groups,
the sample sizes for those groups are often fixed by the sampling
design. We assume a binomial distribution for the sample in
each row, with number of trials equal to the fixed row total.

When the columns are a response variable $Y$ and the rows are
an explanatory variable $X$, it is sensible to divide the cell counts
by the row totals to form conditional distributions on the
response. In doing so, we inherently treat the row totals as fixed
and analyze the data the same way as if the two rows formed
separate samples.

\subsection{Unconditional Design}
When the total sample size n is fixed and we cross classify the
sample on two categorical response variables, the multinomial
distribution is the actual joint distribution over the cells. The
cells of the contingency table are the possible outcomes, and the
cell probabilities are the multinomial parameters.
