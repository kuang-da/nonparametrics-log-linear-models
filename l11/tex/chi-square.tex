\section{Chi-squared Test of Independence}
\begin{itemize}
	\item Null Hypothesis: $X$ and $Y$ are independent.
	\item Test Statistic:
	\[T = \sum_{i=1}^{I}\sum_{j=1}^{J} \frac{(n_{ij} - \frac{n_i n_{.j}}{n})^2}{\frac{n_i n_{.j}}{n}}\]
	\item Reject the null hypothesis if 
	\[T > \chi^2_{(I-1)(J-1), 1 - \alpha}\]
\end{itemize}

\subsection{Example}
Scarlet fever is a childhood infection that among other symptoms
gives rise to severe irritation of the nose, throat and ears. 

In a study, six districts A to F were chosen. In each district,
patients were located, and parents were asked to state the site
at which they thought their child's irritation was worst.

\begin{figure}[H]
	\centering
	\includegraphics[width=0.7\linewidth]{fig/screenshot003}
	\caption{Example of Chi-square Test}
	\label{fig:screenshot004}
\end{figure}


\lstinputlisting[language=R]{code/l11-exp2.R}

Note that there is warning when running the \texttt{chisq.test}. It is because that the chi-square test is based on CLT but the sample size is not large enough for approximation. 

To have enough sample size, for all cells of the contingency table, we have
\[\frac{n_i n_{.j}}{n} > 1 \text{ or } > 5.\]

We turn to Fisher Exact test to have better result. But note that the odd ratio is undefined for multinomial distribution.